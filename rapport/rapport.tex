\documentclass[10pt,oneside]{article}   %% Standard LaTeX.
\usepackage{url}             %% Supports formating URLs.
%% \usepackage{graphicx}        %% Enable for eps figures
\usepackage[utf8]{inputenc}

\begin{document}

\title{Lecture et fusion de fichiers ELF}           %% For first page
\author{Adrian Amaglio\\Mohamed-Karim Boussedra\\Benjamin Gontard\\Maxence Grand\\Quentin Rozand}
\date{\today}
\maketitle
\tableofcontents


\section{Executables produits}
	\begin{minipage}{\textwidth}
		L'executable phase1 comprend toutes les fonctionnalités d'affichage du fichier ELF. 
		Il s'utilise de la meme manière que le programme readelf \cite{readelf}:\\
			\texttt{phase1 [-sShr] [-x <num|nom>] <fichier>}\\
			\texttt{-s} Affiche la table des symboles.\\
			\texttt{-S} Affiche la table des en-tetes de sections.\\
			\texttt{-h} Affiche l'en-tete ELF.\\
			\texttt{-x <num|nom>} Affiche Le contenu de la section numéro \texttt{<num>} ou nommée \texttt{<nom>}, en hexadécimal et en ASCII lorsque c'est possible.\\
			\texttt{-r} Affiche la table des réallocations.\\
	\end{minipage}
	Toutes ces options s'appliquent au fichier passé en dernier paramètre. L'executable phase2 prend trois noms de fichiers en paramètre. Il fusionnera les deux premier dans le troisième.

\section{Structure du code}
	\subsection{Organisation des sources}
	Le code est disponible sur github\cite{github} dans sa version finale. Le code fonctionnel est sur la branche \it{master}
	\begin{description}
		\item [\bf{getelf.h}] Structure elf\_t réprésentant un fichier ELF en mémoire.
		\item [\bf{getelf.c}] Fonctions de lecture d'un fichier ELF et de remplissage de la structure elf\_t. Son "api publique" n'est composée que de la fonction initElf(elf\_t elf, str *filename) qui s'occupe d'ouvrir et charger le fichier.
		\item [\bf{display.c}] Affichage d'informations contenues dans une structure elf\_t.
		\item [\bf{phase1.c}] Lecture d'un fichier ELF avec les fonctions de getelf.c et appel des fonctions de display.c en fonction des arguments.
		\item [\bf{fusionelf.c}] Fonctions de fusion de deux structures elf\_t.
		\item [\bf{writeelf.c}] Écriture d'un fichier ELF à partir d'une structure elf\_t.
		\item [\bf{phase2.c}] Lecture de deux fichiers ELF avec les fonctions de getelf.c, fusion avec fusionelf.c et écriture d'un fichier résultat avec writeelf.
	\end{description}

	\subsection{Principe de fonctionnement}
	Le contenu de chaque fichier ouvert en lecture est copié en mémoire dans une structure organisée. La récupération des information se fait dans ces structures pour l'affichage ou la fusion. Dans l'idéal, l'opération de fusion ne devrait pas écrire dans un fichier mais seulement dans une structure. L'écriture sur le disque serait alors prise en charge par une fonction dédiée.


\section{Fonctionalités implémentées et manquantes}
	Toutes ces fonctionnalités concernent l'affichage et la fusion.
	\subsection{Implémentées}
		\begin{itemize}
			\item Table des symboles
			\item Table des en-tetes de sections,
			\item En-tete elf
			\item Contenu d'une section
			\item Table des réallocations
		\end{itemize}
	\subsection{Manquantes}
		\begin{itemize}
			\item Section des symboles dynamiques
			\item Sections unwind
			\item Sections de notes
			\item En-tetes de programmes
		\end{itemize}

\section{Bogues}
	Le fichier fusionné final, une fois compilé, n'est pas executable.


\section{Tests effectués}
	Un script bash effectue automatiquement une série de tests et écrit le résultat dans des fichiers textes.

\section{Journal de bord}

	\tabcolsep=0.11cm
	\begin{tabular}{| c || *{4}{c|} c |}
		\hline
		& Adrian & Mohamed-Karim & Benjamin & Quentin & Maxence \\
		\hline
		\bf{03/01/17} & \multicolumn{5}{c|}{Lecture doc, Étapes 1, 2, 3, mise en place du dépot git}\\
		\hline
		\bf{04/01/17} & \multicolumn{2}{c|}{Récupération des noms de sections} & \multicolumn{3}{c|}{Suite 1, 2 \& 3} \\
		\hline
		\bf{05/01/17} & \multicolumn{2}{c|}{Ajout de la structure elf\_t} & \multicolumn{3}{c|}{Étapes 4\&5} \\
		\hline
		\bf{06/01/17} & \multicolumn{2}{c|}{Refactorisation de la phase 1} & \multicolumn{3}{c|}{Début de la phase 2} \\
		\hline
		\bf{09/01/17} & \multicolumn{2}{c|}{\parbox[t]{5cm}{Extention de elf\_t\\Récriture des fonction d'affichage}} & \multicolumn{3}{c|}{Etape 6\&7} \\
		\hline
		\bf{10/01/17} & \multicolumn{2}{c|}{récriture des fonction d'affichage} & \multicolumn{3}{c|}{Etape 8\&9} \\
		\hline
		\bf{11/01/17} & \multicolumn{2}{c|}{\parbox[t]{5cm}{Structuration des fichiers\\ Début du rapport}} & \multicolumn{2}{c|}{Correction bug etape 8 et 9} & Tests\\
		\hline
		\bf{12/01/17} & \multicolumn{5}{c|}{\parbox[t]{5cm}{Restructuration du projet\\Travail sur la cohérence du code} }\\
		\hline
	\end{tabular}

\begin{thebibliography}{9}

\bibitem{readelf}
\url{https://linux.die.net/man/1/readelf}

\bibitem{github}
\url{https://github.com/AdrianUGA/fusion}

% \bibitem{flanders}
%  H. Flanders,  {\it Manual for\/ {\rm Monthly} Authors}, Amer. Math.
% Monthly {\bf 78} (1971), 1--10.

\end{thebibliography}


\end{document}
